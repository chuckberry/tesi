Nella sfida per fabbricare la CPU pi\'u performante, i progettisti hardware si devono confrontare con un problema molto complicato. Da un lato la Legge 
di Moore non si pu\'o pi\'u applicare alla capacit\'a computazionale di una macchina, cioe\': la capacit\'a computazionale non si raddoppia pi\'u ogni 18 
mesi, come nel passato. Dall' altro lato, il consumo di potenza continua ad aumentare pi\'u che linearmente con il numero di transistors inclusi in un chip 
e la Legge di Moore vale ancora per il numero di transistor in un chip. Parecchie soluzioni sono state adottate per risolvere questo problema. Alcune die 
esse cercano di ridurre il consumo di potenza sacrificando la capacit\'a computazionale, di solito usando il frequency scaling, il voltage throttle o 
entrambi. Altre soluzioni cercano di in aumentare l' Instruction Level Parallelism (ILP) all'interno di un processore, in modo da avere pi\'u capacit\'a 
computazionale da un processore senza aumentare il consumo di potenza. Tuttavia, oggigiorno il costo di un cache miss (che potrebbe mandare in stallo la 
pipeline) o di uno sbaglio nella predizione di un branch (che potrebbe invalidare la pipeline) sono diventati troppo elevati. Gi\'a nei primi anni 2000, era
chiaro che la via pi\'u efficace per aumentare la capacit\'a computazionale e ridurre il consumo di potenza consisteva nel parallelizzare l' esecuzione dei
task. Per questo motivo \'e stato introdotto il Simultaneous MultiThreading (SMT). Questa tecnologia permetteva di eseguire due thread in modo concorrente 
sulla stessa CPU, in questo modo \'e possibile avere una grande parallelizzazione dell' esecuzione dei task. In base a come sono organizzate le memorie, i
sistemi a multiprocessore sono classificati in due gruppi:

\begin{description}

\item [Centralised Shared Memory Architectures:] in queste architetture, ci sono diversi core connessi ad una memoria condivisa. Se tutti i cores 
sono uguali, queste architetture si chiamano simmetric multiprocessor (SMP).

\item [Distributed Memory Architecture:] in queste architetture, ciascun processore ha il suo proprio modulo di memoria e il tempo di accesso alla memoria 
dipende dalla locazione di memoria relativamente al processore che vi accede. In questa categoria di architetture sono incluse le Non-uniform memory access 
(NUMA).

\end{description}

Le architetture multicore sono state adottate da molte industrie di microprocessori. I chips Dual-core sono ormai una soluzione comune e numerose opzioni da
4 e 8 cores sono disponibili. Negli anni a venire, il numero di core integrati in un chip continuer\'a a crescere: Intel ha annunciato che rilascer\'a per
chips a 80 core per il 2013. Il passaggio alle tecnologie multicore \'e un evento di svolta, poich\'e esso porta le piattaforme di computazione standard ad
essere multiprocessori. 

Anche molti sistemi embedded stanno iniziando ad usare architetture multicore, perch\'e questi processori forniscono un grande incremento di capacit\'a 
computazionale con un basso incremento di consumo di potenza e questo \'e un aspetto importante per questo tipo di dispositivi. Ma c\'e un' ostacolo 
all' uso di queste architetture in nel settore embedded e in particolare nei sistemi Real-Time. Immaginate questa situazione: ci sono 3 tasks Real-time: A, 
B e C. A usa 512KB di memoria, B ne usa 768KB e C ne usa 256KB. La nostra piattaforma \'e un Dual-core provvisto di cache condivisa onchip di 1MB. Ci sono 
due casi possibili di scheduling. Nel primo caso A e C (o B e C) sono schedulati concorrentemente. C'\'e abbastanza spazio in cache per allocare le risore
dei due task, perci\'o non c'\'e nessun problema. Nel secondo caso A \'e schedulato insieme a B: avviene cache thrashing. Le performante dei due task 
potrebbero peggiorare rispetto al caso precedente, perch\'e non c'\'e alcuna garanzia che A e B possano trovare i dati necessari nella cache condivisa e, 
inoltre, \'e impossibile prevedere la durata di A o B, perch\'e se A \'e schedulto insieme a B, esso avr\'a una certa durata. Invece, se A \'e schedulato 
insieme a C, esso avr\'a un' altra durata, in altre parole: la durata di un task dipende da quale altro task \'e schedulato con esso e, per questo motivo, 
utilizzando i comuni algoritmi di scheduling Real-Time, le tecniche di analisi dei tempi per il software embedded usate nei sistemi a 
single core non sono pi\'u utilizzabili nei sistemi multicore. Per questa ragione, sono necessarie nuove tecniche per stimare il worst-case execution time
(WCET) dei tasks Real-Time. 

Risulta chiaro quindi, che lo scheduler gioca un ruolo importante nel migliorare la predicibilit\'a e le performance delle applicazioni. Oggigiorno \'e 
importante sviluppare algoritmi di scheduling "cache-aware", cio\'e uno scheduler che, per scegliere la CPU dove mettere un task, considera come i tasks 
schedulati usano la memoria cache, in modo da evitare il cache thrashing. Questa tesi \'e il proseguimento del lavoro svolto da Lucas De Marchi, egli ha 
provato a rendere lo scheduler Real-Time di Linux cache-aware, introducendo il concetto di task-affinity.

\section{Stato dell'arte}
\label{sec:StateDellArte}

Sebbene il problema di progettare un algoritmo di scheduling cache-aware sia vecchio e ben conosciuto da oltre 20 anni e i multicore siano ampiamente 
diffusi, attualmente i sistemi operativi in commercio non implementano questo tipo di algoritmi e in letteratura sono presenti solo pochi lavori che 
studiano questo problema. I lavori di ricerca pi\'u recenti relativi a questo argomento consistono in attivit\'a di profiling, che hanno lo scopo di 
dimostrare come una condivisione scorretta della cache da parte di task concorrenti possa rallentarli e causare: throughput sub-ottimale, cache thrashing
e, in alcuni casi, task starvation per i task che non riescono ad occupare spazio sufficiente nella cache necessario per fare buoni progressi nella loro 
esecuzione. Il primo lavoro ben documentato relativo a questo tipo di scheduling \'e stato sviluppato all' universit\'a di Stanford. Alla fine degli anni 
80, il Computer Systems Laboratory di Stanford svilupp\'o un prototipo di un multiprocessore a memoria condivisa chiamato DASH. La sua architettura era 
molto simile a quella usata nel moderni processori SMP; DASH poteva incorporare fino a 64 processori RISC. Per poter sfruttare le piene potenzialit\'a della
macchina, venne sviluppato un runtime system ad hoc ed un linguaggio: COOL. Esso era un' estensione del C++ che introduceva alcuni statements per facilitare
la definizione del parallelismo a grana fine o grossa e per definire quali erano i pattern di accesso alla memoria eseguiti dall' applicazione. Il 
compilatore di COOL era in grado di estrarre automaticamente informazioni sul parallelismo del programma che potevano eesere usate da 
architetture che, come DASH, supportavano un tale livello di concorrenza, ed informazioni sull' uso della cache fatto dall' applicazione. Usando queste
informazioni, il runtime system poteve assicurare il parallelismo desiderato dal programmatore e cercare di ridurre il miss rate di ciascun task, perch\'e 
il sistema "sapeva", per ciascun task, quali fossero gli oggetti referenziati da esso; in questo modo schedulava oggetti e tasks in modo da renderli vicini.
In parole povere, usando informazioni aggiuntive fornite dal programmatore e sfruttando il principio di localit\'a dei dati, il runtime system decideva dove
allocare gli oggetti e assegnava ad un task una CPU che conteneva nella sua cache gli oggetti referenziati da esso. Il progetto COOL mostra come un uso 
intelligente della cache sia un problema che coinvolge tutti gli aspetti dell' ingegneria del software, dal compilatore allo scheduler e il sistema di 
gestione della memoria.

Altri Lavori di ricerca svolti in questi anni sfruttano un' altra strategia. Essi non introducono un nuovo linguaggio di programmazione o sofisticati 
ambienti di runtime, ma implementano un profiler grezzo che, a runtime, inferisce quanto spazio nella cache \'e richiesto da un task, in modo da inferire 
quali tasks potrebbero causare cache thrashing, se fossero schedulati concorrentemente. Per fare questo lavoro, il profiler esegue una fase periodica di
taratura in cui analizza il miss rate di ciascun task, in questo modo, \'e possibile capire l' ammontare di spazio usato da un task. Sulla base di queste
informaizoni due o pi\'u tasks sono schedulati su diverse CPUs solo se non causano cache thrashing. Questi lavori non sono efficaci come COOL, ma presentano
buoni risultati con i benchmark presenti nella suite SPEC2000, inoltre alcuni di questi lavori sono stati sprimentati anche in sistemi embedded con esiti
soddisfacenti.

\section{Obiettivi di questa tesi}
\label{sec:ObbiettiviTesi}

Lo scopo principale di questa tesi \'e l' ottimizzazione della versione attuale della task-affinity. In un primo passo, analizzeremo come la task-affinity 
si comporta su diverse architetture multicore; in particolare sull' Intel Xeon E5440 e sull'Intel i7 870. Queste due architetture sono state scelte
perch\'e hanno una architettura delle cache molto diversa tra di loro e inoltre, anche il sistema di comunicazione tra chip \'e molto diverso in termini
di prestazioni. Con l'analisi che si vuole eseguire, cercheremo di capire quali sono gli aspetti dell'attuale logica della task-affinity da migliorare per
poter sfruttare al meglio le architetture testate, per esempio: attualmente, come descritto in \cite{lcs}, la politica di migration effettuata non \'e 
molto efficace, perch\'e alcuni task tendono a rimbalzare da una CPU all'altra, ad ogni iterazione del benchmark. Riteniamo che questo fenomeno degradi
le peformance ottenibili con la task-affinity, perch\'e spesso un task che migra deve anche eseguire il warm up della cache della CPU su cui esso \'e
migrato, e questo aumenta il miss rate. Quello che ci aspettiamo \'e che l'analisi metta in luce, per ogni architettura, quanto questo "migration pattern" 
incida sul miss rate di un task.

Tenendo conto dei risultati presentati nella fase di analisi, proporremo un' ottimizzazione volta a migliorare la localit\'a temporale dei dati, in modo da
diminuire il miss rate di un task; per fare ci\'o includeremo nella logica della task-affinity anche le funzioni usate per la migrazione dei task. 
Nell'ultima parte dell' ottimizzazione introdurremo la sincronizzazione per le strutture dati usate nella task-affinity. Tutte le misure effettuate sulla 
task-affinity e sul vanilla sono state eseguite usando il benchmark usato in \cite{lcs}.

Quindi gli obbiettivi di questa tesi possono essere riassunti come:

\begin{enumerate}
\item analizzare il comportamento dell'attuale versione della task-affinity su diverse archietture per capire quali aspetti migliorare.
\item ottimizzare la versione attuale della task-affinity, migliorando la politica di migrazione dei task e migliorando la localit\'a temporale dei dati 
garantita dalla task-affinity.
\end{enumerate}

Tutte le patch sviluppate in questa tesi si basano sulla versione 2.6.34 del kernel Linux 

\section{Organizzazione della tesi}
\label{sec:OrganizzazioneTesi}

\begin{description}

\item [Chapter 2:] Nella prima sezione del capitolo, discutiamo i problemi derivanti da un'uso sbagliato della cache. Vedremo come un'uso scorretto della 
cache possa degradare le performance e ridurre notevolmente il determinismo delle applicazioni. Nella sezione successiva, abbiamo effettuato uno studio 
sulle diverse architetture della cache focalizzandoci su quei dettagli architetturali che spesso non sono ben documentati, come: i protocolli di coerenza, 
cache inclusive o esclusive ecc. Nell'ultima sezione, \'e presente una classificazione degli algoritmi di scheduling cache aware sviluppati negli ultimi 
anni, analizzando vantaggi e svantaggi di ciascun algoritmo presentato.  

\item [In Chapter 3:] Nella prima sezione analizziamo l'ottimizzazione implementata in questa tesi. La prima sezione analizza il comportamento della 
task-affinity su diverse architetture, in modo da capire come la task-affinity possa essere ottimizzata. La sezione successiva tratta la sua implementazione
in Linux.

\item [Chapter 4:] presenta i risultati sperimentali che riguardano: la correttezza della soluzione, cio\'e se gli scheduling eseguiti sono approssimabili 
con gli scheduling ideali definiti all'inizio del capitolo, e i miglioramenti rispetto alla versione attuale della task-affinity.

\item [In Chapter 5:] illustreremo le conclusioni del lavoro svolto, riassumendo quali risultati sono stati raggiunti e possibili sviluppi futuri.

\end{description}
